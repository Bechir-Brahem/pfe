
\setcounter{chapter}{2}
\chapter{Implementation}
\minitoc %insert la minitoc
\graphicspath{{Chapitre3/figures/}}

%\DoPToC
%==============================================================================
\pagestyle{fancy}
\fancyhf{}
\fancyhead[R]{\bfseries\rightmark}
\fancyfoot[R]{\thepage}
\renewcommand{\headrulewidth}{0.5pt}
\renewcommand{\footrulewidth}{0pt}
\renewcommand{\chaptermark}[1]{\markboth{{\chaptername~\thechapter. #1 }}{}}
\renewcommand{\sectionmark}[1]{\markright{\thechapter.\thesection~ #1}}

\begin{spacing}{1.2}

%==============================================================================
\section*{Introduction}
Ce chapitre porte sur la partie pratique ainsi que la bibliographie.

\section{Outils et langages utilisés}
L'étude technique peut se trouver dans cette partie, comme elle peut être faite en
parallèle avec l'étude théorique (comme le suggère le modèle 2TUP).
Dans cette partie, il faut essayer de convaincre le lecteur de vos choix en termes de
technologie. Un état de l'art est souhaité ici, avec un comparatif, une synthèse et un choix 
d'outils, même très brefs.
\section{Présentation de l'application}
Il est tout à fait normal que tout le monde attende cette partie pour coller à souhait toutes les images
correspondant aux interfaces diverses de l'application si chère à votre coeur, mais
abstenez vous! Il FAUT mettre des imprime écrans, mais bien choisis, et surtout, il faut les scénariser : Choisissez un scénario d'exécution, par exemple la création d'un 
nouveau client, et montrer les différentes interfaces nécessaires pour le faire, en
expliquant brièvement le comportement de l'application. Pas trop d'images, ni trop de
commentaires : concis, encore et toujours.

Évitez ici de coller du code : personne n'a envie de voir le contenu de vos classes.
Mais  vous  pouvez insérer des snippets (bouts de code) pour montrer certaines
fonctionnalités \cite{knuth-acp}\cite{dirac}, si vous en avez vraiment besoin. Si vous voulez montrer une partie de votre code, les étapes d'installation ou de configuration, vous pourrez les mettre dans l'annexe.
\subsection{Exemple de tableau}

Vous pouvez utiliser une description tabulaire d'une éventuelle comparaison entre les travaux existants. Ceci est un exemple de tableau: Tab \ref{tab:exple}.

\begin{table}[ht]
	\centering
	\caption{Tableau comparatif}
	\footnotesize
	\begin{tabularx}{\linewidth}{|>{\bfseries \vspace*{\fill}}X ||>{\centering{}\vspace*{\fill}}X|>{\centering{}\vspace*{\fill}}X|>{\centering{}\vspace*{\fill}}X|>{\vspace*{\fill}}X<{\centering{}}|}	
			\hline 
			& \bfseries Col1 & \bfseries Col2 &\bfseries Col3 &\bfseries Col4\\
			\hline \hline
			Row1		&		&	X	&		&		\\
			Row2		&	X	&		&		&		\\
			Row3		&	X	&	X	&	X	&	X	\\
			Row4		&	X	&		&	X	&	X	\\
			Row5		&	X	&		&	X	&	X	\\
			Row6		&	X	&		&	X	&	X	\\
			Row7		&	X	&		&	X	&		\\
			Row8		&	X	&	X	&	X	&		\\
			\hline
	\end{tabularx}
	\label{tab:exple}
\end{table}

\subsection{Exemple de Code}
Voici un exemple de code Java, avec coloration syntaxique \ref{code:java}.

\begin{lstlisting}[rulecolor=\color{white}]
\end{lstlisting}

\begin{lstlisting}[label=code:java,caption=Helloworld Java,language=java]
	public class HelloWorld {
    // comment
    public static void main(String[] args) {
        System.out.println("Hello, World");
    }

}
\end{lstlisting}

\section{Remarques sur la bibliographie}
Votre bibliographie doit répondre à certains critères, sinon, on vous fera encore et
toujours la remarque dessus (et parfois, même si vous pensez avoir tout fait comme il
 faut, on peut vous faire la remarque quand même : chacun a une conception très
personnelle de comment une bibliographie devrait être).\\
\begin{itemize}
\item Une bibliographie dans un bon rapport doit contenir plus de livres et d'articles 
que de sites web : après tout c'est une biblio. Privilégiez donc les ouvrages
reconnus et publiés pour vos définitions, au lieu de sauter directement sur le premier article wikipedia;
 \item Les éléments d'une bibliographie sont de préférence classés par ordre
alphabétique, ou par thèmes (et ordre alphabétique pour chaque thème);
\item Une entrée bibliographique doit être sous la forme suivante :
\begin{itemize}
\item Elle doit contenir un identifiant unique: représenté soit par un numéro
[1] ou par le nom du premier auteur, suivi de l'année d'édition [Kuntz, 1987];
\item Si c'est un livre : Les noms des auteurs, suivi du titre du livre, de l'éditeur, 
ISBN/ISSN, et la date d'édition;
\item Si c'est un article : Les noms des auteurs, le titre , le journal ou la
conférence, et la date de publication;
\item Si c'est un site web ou un document électronique : Le titre, le lien et la date 
de consultation;
\item Si c'est une thèse : nom et prénom, titre de la thèse, université de
soutenance, année de soutenance, nombre de pages;
\item Exemples : 
\begin{description}
\item $[Bazin, 1992]$ BAZIN R., REGNIER B. Les traitements antiviraux et leurs essais
thérapeutiques. Rev. Prat., 1992, 42, 2, p.148-153.\\
\item $[Anderson,1998]$ ANDERSON P.JF. Checklist of criteria used for evaluation of metasites.
[en ligne]. Université du Michigan, Etats Unis. Site disponible sur :\\
http://www.lib.umich.edu/megasite/critlist.html.(Page consultée le 11/09/1998).
\end{description}
\item Dans le texte du rapport, on doit obligatoirement citer la référence en  faisant appel à son identifiant, juste après avoir utilisé la citation. Si ceci n'est pas fait dans les règles, on peut être accusé de plagiat.
\end{itemize} 
\end{itemize} 

\section*{Conclusion}
Voilà.

%==============================================================================
\end{spacing}
