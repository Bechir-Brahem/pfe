
\setcounter{chapter}{1}
\chapter{Requirement Specification and Overall Architecture}
\minitoc %insert la minitoc
\graphicspath{{Chapitre2/figures/}}

%\DoPToC

%==============================================================================
\pagestyle{fancy}
\fancyhf{}
\fancyhead[R]{\bfseries\rightmark}
\fancyfoot[R]{\thepage}
\renewcommand{\headrulewidth}{0.5pt}
\renewcommand{\footrulewidth}{0pt}
\renewcommand{\chaptermark}[1]{\markboth{\MakeUppercase{\chaptername~\thechapter. #1 }}{}}
\renewcommand{\sectionmark}[1]{\markright{\thechapter.\thesection~ #1}}

\begin{spacing}{1.2}
    %==============================================================================
    \section*{Introduction}
    The requirement specification is the first step in the software development process.
    It is the basis for the design and implementation of the software system.
    It is the process of defining, documenting and maintaining the requirements of the system.
    The requirements are the description of the system services and constraints that are to be implemented.

    In this chapter we will present the requirement specification of the project, we will define our actors,
    the functional and non-functional requirements of the system.
    We will also present the overall architecture of the project and the design principles that we will follow.

    \section{Requirement Specification}

    \subsection{Actors Identification}
    Our library is developed for one main actor: the scientist.
    The scientist is the person who is going to use the library to develop new data processing algorithms,
    acquire data from different IO sources, analyze the data and visualize the results.
    It is assumed that the scientist is very competent but is not necessarily an expert in software development.
    The library is designed to be easy to use and to provide a high level of abstraction to the scientist.



    \subsection{Functional Specification}
    \subsection{Non-Functional Specification}
    \section{Overall Architecture and Guidelines}
    \subsection{Onion Architecture}
    \subsection{SOLID Principles}
    \subsection{C++ Specific Design}
    \subsubsection{Templates Metaprogramming}
    \subsubsection{C++ Idioms}
    \subsubsection{C++ Limitations}
    \subsection{Project Architectural Design}
    \subsubsection{Project Modules}
    \subsubsection{Project Architecture}
    \subsubsection{Class Diagram}
    % \begin{figure}[!ht]\centering
    % \includegraphics[scale=0.9]{stereotypes.jpg}
    % \caption{Les stéréotypes}
    % \label{fig:fig2}
    % \end{figure}


    \section*{Conclusion}
    Faire ici une petite récapitulation du chapitre, ainsi qu'une introduction du chapitre suivant.





    %==============================================================================
\end{spacing}
