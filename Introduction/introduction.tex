\chapter*{General Introduction}

\addcontentsline{toc}{chapter}{General Introduction}
\begin{spacing}{1.2}
%==================================================================================================%



% \textbf{Une introduction} doit être rédigée sous forme de paragraphes bien ficelés. Elle est
% normalement constituée de 4 grandes parties :
% \begin{enumerate}
% \item Le contexte de votre application : le domaine en général, par exemple le domaine du web, de BI, des logiciels de gestion ?
% \item La problématique : quels sont les besoins qui, dans ce contexte là, nécessitent la réalisation de votre projet?
% \item La contribution : expliquer assez brièvement en quoi consiste votre application, sans entrer dans les détails de réalisation. Ne pas oublier qu'une introduction est
%  censée introduire le travail, pas le résumer; 
%  \item La composition du rapport : les différents chapitres et leur composition. Il n'est pas nécessaire de numéroter ces parties, mais les mettre plutôt sous forme de paragraphes successifs bien liés.
% \end{enumerate}

As the Aare river flows across Switzerland, it passes from Bern to Aargau and encounters in 
its path the Paul Scherrer Institute (PSI). This is where science meets technology to provide solutions
to the world's most pressing challenges. In this research institute, thousands of scientists
and engineers work together leveraging the power of particle accelerators to study the
natural world and develop new technologies. \\

Particle accelerators are an engineering and scientific marvel. They are used in a wide range of
applications, from fundamental research in particle physics to medical treatments and industrial
applications. As these large scale research instruments become more advanced and complex, the data they produce
increases in volume and velocity.\\

This is where the need for a flexible and high performance data analysis library arises. The Aare
project aims to provide such a library for the analysis of data produced by hybrid pixel detectors.
Scientists at PSI can rely on Aare to process and analyze the data produced by these detectors,
all while abstracting away the complexities of the underlying hardware and software.\\

This report presents the work done during the development of the Aare library. It is divided into 
four chapters as follows. Chapter 1 provides an overview of the project and the context in which
it was developed. Chapter 2 presents the requirements, the architectural design, and the 
overall guidelines for the project. Chapter 3 details the implementation of the library,
including the technologies used and the challenges faced. Finally, Chapter 4 presents the
results of the project and evaluates the performance of the library.\\








\end{spacing}


