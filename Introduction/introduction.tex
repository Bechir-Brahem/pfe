\chapter*{General Introduction}

\addcontentsline{toc}{chapter}{General Introduction}
\begin{spacing}{1.2}
%==================================================================================================%



% \textbf{Une introduction} doit être rédigée sous forme de paragraphes bien ficelés. Elle est
% normalement constituée de 4 grandes parties :
% \begin{enumerate}
% \item Le contexte de votre application : le domaine en général, par exemple le domaine du web, de BI, des logiciels de gestion ?
% \item La problématique : quels sont les besoins qui, dans ce contexte là, nécessitent la réalisation de votre projet?
% \item La contribution : expliquer assez brièvement en quoi consiste votre application, sans entrer dans les détails de réalisation. Ne pas oublier qu'une introduction est
%  censée introduire le travail, pas le résumer; 
%  \item La composition du rapport : les différents chapitres et leur composition. Il n'est pas nécessaire de numéroter ces parties, mais les mettre plutôt sous forme de paragraphes successifs bien liés.
% \end{enumerate}

--TODO--






\end{spacing}


