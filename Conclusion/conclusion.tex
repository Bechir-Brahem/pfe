\chapter{Conclusion and Perspectives}
%==============================================================================
\pagestyle{fancy}
\fancyhf{}
\fancyhead[R]{\bfseries\rightmark}
\fancyfoot[R]{\thepage}
\renewcommand{\headrulewidth}{0.5pt}
\renewcommand{\footrulewidth}{0pt}
\renewcommand{\chaptermark}[1]{\markboth{\MakeUppercase{\chaptername~\thechapter. #1 }}{}}
\renewcommand{\sectionmark}[1]{\markright{\thechapter.\thesection~ #1}}

\begin{spacing}{1.2}
%==============================================================================

C'est l'une des parties les plus importantes et pourtant les plus négligées 
du rapport. Ce qu'on \underline{ne veut pas voir} ici, c'est combien ce stage vous a été bénéfique, comment il vous a appris à vous intégrer, à connaître le monde du travail, etc.\\
Franchement, personne n'en a rien à faire, du moins dans cette partie. Pour cela, vous 
avez les remerciements et les dédicaces, vous pourrez vous y exprimer à souhait.\\
La conclusion, c'est très simple : c'est d'abord le résumé de ce que vous avez raconté
dans le rapport : vous reprenez votre contribution, en y ajoutant ici les outils que vous 
avez utilisé, votre manière de procéder. Vous pouvez même mettre les difficultés
rencontrées. En deuxième lieu, on y met les perspectives du travail : ce qu'on pourrait 
ajouter à votre application, comment on pourrait l'améliorer.

%==============================================================================
\end{spacing}