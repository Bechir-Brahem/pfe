\chapter{Conclusion and Perspectives}
%==============================================================================
\pagestyle{fancy}
\fancyhf{}
\fancyhead[R]{\bfseries\rightmark}
\fancyfoot[R]{\thepage}
\renewcommand{\headrulewidth}{0.5pt}
\renewcommand{\footrulewidth}{0pt}
\renewcommand{\chaptermark}[1]{\markboth{\MakeUppercase{\chaptername~\thechapter. #1 }}{}}
\renewcommand{\sectionmark}[1]{\markright{\thechapter.\thesection~ #1}}

\begin{spacing}{1.2}
%==============================================================================
\section*{Conclusion}
In this thesis, we have presented the requirements, design, implementation and evaluation 
of a flexible data analysis library for pixel detectors. Managing the performance, 
flexibility, extensibility and usability of the library was the main goal of this work.
Several design choices were made to balance the trade-offs between these requirements.\\

In result of this work, a modular design was conceived and implemented. The library 
provides a set of tools for data analysis, file manipulation, network communication,
and parallelization that scientist can use. In summary the library can:

\begin{itemize}
    \item Read and write data in built-in or custom user-defined formats. 
    \item Process incoming data in real-time or offline.
    \item Communicate over the network with other devices or software.
    \item Distribute the data processing over multiple cores or nodes.
    \item Expose a Python API with C++ like performance.
    \item Be extended with new modules or algorithms.
\end{itemize}

In conclusion, aare provides a rich set of tools for scientists to analyze data from
hybrid pixel detectors. It offers abstractions from the underlying infrastructure, 
allowing the scientists to focus on science rather than the technical details.\\


\section*{Lessons Learned}
The development of aare was a challenging task that required a lot of effort and time.
I can confidently say that I have learned a lot during this project. From the architectural 
design, C++ programming, Python bindings, to the communication skills this project has
broadened my vision and knowledge.\\

Several lessons can be presented here but the most important ones are:

\textbf{Over Engineering.} "Any problem in computer science can be solved with another
level of indirection, except for too many levels of indirection." This quote is a good
reminder that over-engineering can be a problem. The first design of the file\_io 
included a lot design patterns and abstractions that were not necessary. This impacted 
the development time and increased the complexity of the code.

A better approach would have been to start with a simple design and then refactor the
code as needed. Projects evolve and requirements change, making prototypes, iterating 
over them and discussing with the users is a better approach.\\

\textbf{C++ bindings: Python on steroids.} The Python bindings were a challenge to implement. But, 
the performance of the Python API was a pleasant surprise. CPython is interoperable with C++
and moving data between the two languages is efficient. Using features such as the buffer
protocol, NumPy arrays or other objects can be shared between C++ and Python without 
copying the data. 

Slow Python code can be rewritten in C++ and exposed to Python with a simple interface.
This simple trick allows aare to achieve its goals of flexibility, performance and usability.
Furthermore, the controversial GIL limitation can be worked around even with multiples threads.
This permits object sharing between threads and parallelization\\

\textbf{Architecture and Design.} The architecture of aare was a key factor in the success of the project.
The modular design allowed the library to be extended with new modules and manage 
the dependencies between them more easily. Dependencies are a risk in software development
and it can slow down development and increase the complexity of the code. 

Spending time on the design and architecture of the library was a good investment. It should be 
noted that the design only specified the high-level components and their interactions. 

\textbf{Tests.} Writing tests was a good practice that helped to catch bugs early in the
development process. Aare was developed a component by component, starting with the core components
and then adding new features. Code evolves and changes will be made. After coming 
back to the codebase after a few weeks, having reliable tests was a good way to ensure that
the changes did not break the existing code.

On the other hand, aare was an early stage, fast changing project. Writing tests very strict 
tests and covering every edge case can negatively impact the development time. While the 
library was in beta stage, it was helpful to leave some parts of the code loosely tested.\\

\section*{Perspectives}
The development of aare is not finished. There are several features that could be added
to the library to make it more useful for scientists. One main area to explore is GPU 
acceleration. The library could be extended to support CUDA or OpenCL to take advantage
of the parallelism of the GPU. This can help for processing data faster and in real-time.\\

Another important point it the need for more users and feedback. The library was developed
with the help of a few users and their feedback was very valuable. More users can help to
identify the missing features, bugs, or improvements that can be made. Tutorials, examples 
and documentation can help to attract more users.\\










%==============================================================================
\end{spacing}