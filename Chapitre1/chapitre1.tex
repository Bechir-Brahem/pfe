\setcounter{mtc}{4} %indique le numéro réel du chapitre, pour la mini table des matières
\setcounter{chapter}{0}
\chapter{Project Context and Scope}
\minitoc  %insert la minitoc

\graphicspath{{Chapitre1/figures/}}
%==============================================================================
\pagestyle{fancy}
\fancyhf{}
\fancyhead[R]{\bfseries\chaptername~\thechapter. }
\fancyfoot[R]{\thepage}
\renewcommand{\headrulewidth}{0.5pt}
\renewcommand{\footrulewidth}{0pt}
%\renewcommand{\chaptermark}[1]{\markright{\MakeUppercase{\chaptername~\thechapter. #1 }}{}}
%\renewcommand{\sectionmark}[1]{\markright{\thechapter.\thesection~ #1}}

\begin{spacing}{1.2}
%==============================================================================

\section*{Introduction}
Une étude théorique \cite{knuthwebsite} peut contenir l'une et/ou l'autre de ces deux parties :
\section{Presentation of The Host Company} 
% \begin{figure}[!ht]\centering
% \includegraphics[scale=0.9]{art.jpg}
% \caption{État de l'art}
% \label{fig:fig1}
% \end{figure}
\section{Detector's Group Presentation and Work}
\section{Problem Statement}
\section{Project Goals}
\section{Work Methodology}
\section{High Level Planning}

\section*{Conclusion}
La conclusion est en général sans numérotation, et n'apparaît pas dans la table des matières.


%==============================================================================
\end{spacing}
