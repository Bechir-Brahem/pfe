\setcounter{mtc}{4} %indique le numéro réel du chapitre, pour la mini table des matières
\setcounter{chapter}{0}
\chapter{Project Context and Scope}
\minitoc  %insert la minitoc

\graphicspath{{Chapitre1/figures/}}
%==============================================================================
\pagestyle{fancy}
\fancyhf{}
\fancyhead[R]{\bfseries\chaptername~\thechapter. }
\fancyfoot[R]{\thepage}
\renewcommand{\headrulewidth}{0.5pt}
\renewcommand{\footrulewidth}{0pt}
%\renewcommand{\chaptermark}[1]{\markright{\MakeUppercase{\chaptername~\thechapter. #1 }}{}}
%\renewcommand{\sectionmark}[1]{\markright{\thechapter.\thesection~ #1}}

\begin{spacing}{1.2}
    %==============================================================================

    \section*{Introduction}
    In this first chapter, The Paul Scherrer institute and its detectors group are introduced.
    the need for our project and the problems it aims to solve will be explained. Furthermore,
    I will go through the goals of the project and provide a high-level timeline
    capturing its progression.

    \section{Presentation of The Host Company}
    The Paul Scherrer Institute (PSI) is the largest research institute for natural and engineering sciences within Switzerland.
    Created in 1998, the institute is located in Canton Aargau and employs around 2300 people.
    PSI is composed of 8 main research centers:
    \begin{itemize}
        \item Center for Life Sciences
        \item Center for Neutron and Muon Sciences
        \item Center for Nuclear Engineering and Sciences
        \item Center for Energy and Environmental Sciences
        \item Center for Photon Science
        \item Center for Scientific Computing
        \item Theory and Data
        \item Center for Accelerator Science and Engineering.
    \end{itemize}

    The Paul Scherrer Institute operates several large scale research facilities including two
    X-ray light sources: the Swiss Light Source (SLS) and SwissFEL (Swiss Free Electron Laser).\\


    SwissFEL is a free electron laser
    that provides ultra-short and high-intensity x-ray pulses. With their unrivalled brightness
    and short pulse duration, Free Electron Lasers (FEL) support progress
    towards the development of faster and smaller magnetic storage devices,
    a better understanding of catalytic materials for chemistry, bring along dramatically
    improved imaging techniques for bio-molecules in drug discovery and provide new
    insights into poorly understood materials that are technologically appealing. \cite{whySwissFEL}\\


    The SLS synchotron is a charged particle accelerator that accelerates electrons
    to nearly the speed of light. The electrons are then
    forced to travel in a circular path, and emit synchrotron radiation in the form of x-rays.
    Scientists use the x-rays to study the properties of materials, and to perform experiments in a wide range of fields.\\

    The SLS is a third-generation synchrotron light source, which provides high-brilliance photon beams
    with high spectral resolution and tunable energy. It is used for reseach in materials science,
    biology, chemistry and more. \cite{boge2002first, aboutSLS, PhysRevLett.128.024801}.\\

    The SLS is composed of 16 beamlines, each of which is used for a specific type of experiment.
    A beamline consists of an optical hutch where synchotron light is focused and
    wavelength selected. Scientists operate the beamline and perform their experiments
    from the endstation \cite{lightsource}.\\

    In addition, the SLS is undergoing a major upgrade to become a 4-th generation light source.
    SLS 2.0 will provide a higher brilliance    and coherence of the x-ray beams. It will also
    increase the data rate by up to four orders of magnitude \cite{sls2}. This highlights the need for
    new and improved software to handle the increased data rate and complexity of the experiments.\\







    \begin{figure}
        \centering
        \includegraphics[width=\textwidth]{Chapitre1/figures/psi.jpg}
        \caption{A picture of the Paul Scherrer Institute. The SLS is the large circular building.
            PSI has two main parts the East and West separated by the Aare river.}
        \label{fig:sls}
    \end{figure}

    \section{Detector's Group Presentation and Work}
    The Detector's Group is one of the research groups at the Paul Scherrer Institute. It is part of the Laboratory for X-ray Nanoscience and Technologies (LXN) which itself
    is part of the Center for Photon Science.\\

    % The group is composed of around 28 people, including the group leaders.
    The group develops custom high performance detectors not available from commercial vendors and handle everything
    from chip design in the readout chip to FPGA, mechanics and software.\\

    These Detector's are an integral part of the beamline's setup, and are used to detect the x-ray photons or electrons. Many experiments in different
    fields can use these detectors such as for studying the properties of
    materials \cite{butcher2024ptychographic},
    protein structures \cite{pomeranz2009crystal}, crystallography \cite{leonarski2023kilohertz}, biology \cite{lemcoff2023brilliant,dullin2024vivo}, and many other fields of research.


    The detectors stream data at very high rates. For example the JUNGFRAU 4-megapixel (4M)
    charge-integrating pixel-array detector, when operated at a full 2 kHz frame rate,
    streams data at a rate of 17GB/s. MATTERHORN, the newest detector developed by the group,
    can stream data at a rate of 100GB/s \cite{matterhorn, Jungfraujoch}.\\

    On the software side, the group is responsible for the development of the software that is used to control the detectors, acquire the data, and analyze it.
    The software is developed in C++ with python bindings and is used by scientists to perform their experiments.
    The main package maintained by the detector's group is the "SLS Detector Package" available publicly
    on \url{https://github.com/slsdetectorgroup/slsDetectorPackage}. The package provides several binaries such as:
    \begin{itemize}
        \item \textbf{slsReceiver} The receiver server acquires incoming data from detectors using UDP and listens for configurations from host machine using TCP.
        \item \textbf{sls\_detector\_get} Used by the host machine to request the configuration on the Receiver or the Detector.
        \item \textbf{sls\_detector\_put} Used to configure parameters on both the Receiver and Detector.
        \item \textbf{slsDetectorGui} A graphical user interface (GUI) implemented in Qt that receives data from the Receiver and displays it.
    \end{itemize}
    The list of binaries is not exhaustive, and the package contains many other binaries for simulating detectors, analyzing data, and many other utilities.

    \begin{figure}[h]
        \centering
        \includegraphics[width=0.8\textwidth]{Chapitre1/figures/slsreceiver.png}
        \caption{slsDetectorPackage setup for two detectors. Configuration uses TCP while
            data streaming from detector to slsReceiver uses UDP. The data transfer from the detector
            comes over 10GB/s optical fibers.}
        \label{fig:detector}
    \end{figure}

    \section{Problem Statement}
    \subsection{Existing Solution}
    The detector's group includes scientists, software engineers, firmware engineers, chip designers and many more roles.
    The group is diverse and the libraries' usage differs from one user to another. In general the usage of the libraries includes: acquiring data from network
    , configuring receivers and detectors, storing incoming data, processing data on the fly or after storing it.
    For the standard functionalities users rely on the slsDetectorPackage binaries. But the slsDetectorPackage is a generic software
    developed for public use and has very broad functionalities. Hence, for specific use cases scientists might need to write their own scripts
    or change the slsDetectorPackage source code and build again.
    \subsection{Limits of The Existing Solution}
    The slsDetectorPackage is a very powerful software package, but it has some major limitations.
    \subsubsection{Code Complexity}
    First, the code is very complex and has a steep learning curve. The code is written in C++ and uses
    many advanced features of the language. The code is also very large, with around 200 thousand lines of code.
    This makes it difficult for new users to understand how the code works, and to modify it to suit their needs.
    In addition, scientists are not software engineers, and in case of a bug, new feature or a specific use case
    they will be exposed to complex code that they are not familiar with. This might include the need to understand
    C++ code, multi-threading, network programming, and many other advanced topics.


    \subsubsection{Code Rigidity}
    Second, the code is very rigid and inflexible. The code is designed to work in a specific way, and it is difficult
    to modify it to work in a different way. This means that scientists are limited in what they can do with the code,
    and they are forced to work within the constraints of the existing code. This can be very frustrating for scientists,
    who may want to experiment with novel approaches and innovate new methods.
    Furthermore, some of the speicific use case implementations are very brittle and can break easily if the code is modified.
    It lacks proper testing, and logging. This makes it difficult to debug, maintain and extend the code.


    \subsubsection{Code Duplication}
    Third, the code is duplicated in many places. Scientist often rely on their own scripts to
    perform specific tasks.    This leads to each scientist having their own version of the
    code, which is difficult to maintain and update.    and also results in the use of
    sub-optimal code, which is not efficient or reliable. This is due in part to the lack of
    flexibility of the existing code, which forces users to reimplment the same functionalities
    in their own scripts.


    \subsubsection{Data Storage Limitations}
    As the detectors become more and more advanced, the amount of data that they produce is increasing.
    This has made processing the incoming data in real-time a must. The slsDetectorPackage provides limited
    functionalities for processing data on the fly. This means that scientists need to store the data on disk
    and process it later. This is not ideal and is becoming less practical as the amount of data can be very expensive to
    store and process. The new library should be designed to process around 10GB/s of data in real-time.

    % \subsubsection{Strict Stability Requirements}
    % The slsDetectorPackage is used publicly by various groups and institutions.
    % It is tailored for stability and follows a strict release cycle. This means that new features or bug fixes
    % might take a long time to be implemented. 


    \section{Project Goals}
    The goal of this project is to develop a new library that will address the limitations of the existing software.
    The new software package will be designed to be simple, flexible, and efficient. It will be easy to use, and will be
    designed to meet the needs of scientists and engineers. \\

    The library should include functionalities for acquiring data from receiver servers, stream data to receivers,
    read and write raw data files and numpy files, includes commonly used algorithms for data processing and it should
    expose a C++ and a Python API. \\

    In addition, code should be well tested, documented, and should include proper error handling.
    The library should be designed to be extensible, so that new features can be added easily in the future.
    and also flexible so that it accomodates different and upcoming use cases. \\

    The library should be designed to be efficient, so that it can process data in real-time, and should be able to handle
    large amounts of data. It should use parallelism to distribute the load on multiple cores, and should be able to
    take advantage of the GPU for processing data. On the other hand it should abstract the low level details of the
    hardware and network communication to make it easy to use for the scientists.




    \section{Work Methodology}
    We used the \textbf{Kanban} methodology to manage the project. Kanban is a subsystem of the Toyota Production System (TPS),
    which was created in 1940s to control inventory levels, the production and
    supply of components, and in some cases, raw material. \cite{junior2010variations}

    The board is divided into several columns:
    \begin{itemize}
        \item \textbf{Backlog} Contains all the tasks that need to be done.
        \item \textbf{To Do} Contains the tasks that are ready to be worked on.
        \item \textbf{In Progress} Contains the tasks that are currently being worked on.
        \item \textbf{Done} Contains the tasks that are completed.
    \end{itemize}

    \subsection{Tools}
    We used Github Projects to manage the Kanban board. Which is a tool that allows you to create a Kanban board
    and manage your tasks. It is integrated with Github, so you can link your tasks to your code, and track your progress
    easily. We also used Github Issues to create tasks, and Github Pull Requests to review and merge the code.

    The Kanban boards were available for all the group members (involved in project or not), so that they can see the progress
    of the project, and contribute to it if needed. The boards were updated regularly, and the progress was tracked using the
    boards. The boards were also used to plan the work, and to assign tasks to the group members.

    \subsection{Benefits of Kanban}
    The Kanban methodology has several benefits:
    \begin{itemize}
        \item \textbf{Visibility} The Kanban board provides a visual representation of the work
              that needs to be done, and the progress that has been made.
        \item \textbf{Flexibility} The Kanban board is flexible, and can be easily adapted to
              the needs of the project.
        \item \textbf{Efficiency} The Kanban board helps to prioritize the work, and to focus
              on the most important tasks.
        \item \textbf{Collaboration} The Kanban board is a collaborative tool, and can
              be used by all the group members to track the progress of the project.

    \end{itemize}

    \subsection{Meetings}
    We had regular meetings with the group members to discuss the progress of the project, and to plan the work.
    On each Tuesday, The whole group meets for about an hour to discuss the progress of the multiple projects that are being worked on.
    This meeting helps to keep everyone informed about the progress of the projects, and to identify any issues that need to be addressed.

    In addition, on each Friday, a one-on-one meeting is held with the project supervisor to discuss the progress made during the week,
    and to plan the work for the next week. This is a more detailed meeting where we discuss the tasks that need to be done, and the
    kep track of the progress of the project.

    Furthermore, the group has an open door policy, where anyone can ask for help, or discuss any issues that they are facing.
    Knowledge sharing is encouraged, and regular short discussions help overcoming the roadblocks that one might encounter.
    \section{High Level Planning}
    Even though the work methodology is highly flexible, we have a high level planning that we follow.
    The project is divided into several phases:
    \begin{itemize}
        \item \textbf{Phase 1: Research and Project Setup} In this phase, we researched the existing solutions, identified the limitations of the existing software,
              setup the project, and developed a high level architecture for the new library.
        \item \textbf{Phase 2: Implementation of the File Module} In this phase, we implemented the file IO module, which is responsible for reading and writing raw data files and numpy files.
        \item \textbf{Phase 3: Implementation of the Network Module} In this phase, we implemented the network module, which is responsible for acquiring data from receiver servers, and streaming data.
        \item \textbf{Phase 4: Implementation of the Processing Module} In this phase, we implemented the processing module, which includes commonly used algorithms for data processing.
        \item \textbf{Phase 5: Implementation of the Python API} In this phase, we implemented the Python API, which allows users to use the library from Python.
        \item \textbf{Phase 6: Documentation, Testing and Evaluation} In this phase, we tested the library more thoroughly, evaluated the performance, documented the code and added tutorials.
    \end{itemize}

    It is important to note that the phases are not fixed, and can be adapted to the needs of the project. The phases are used to plan the work, and to track the progress of the project.
    For example the python API was implemented in parallel with the other modules, as it was very useful to test the library and to get feedback from the users.

    \begin{figure}[h]\centering
        \includegraphics[width=\textwidth]{Chapitre1/figures/gantt.png}
        \caption{Gantt Diagram of the 6 phases of development. A margin was left at the end of the project to account for
            holidays, vacation days and development delays}
        \label{fig:gantt}
    \end{figure}

    \section*{Conclusion}
    In this chapter we presented the host company and the key difficulties faced
    by its detectors group. Following this we established the goals of the project
    and the methodology that we will follow to achieve them.


    %==============================================================================
\end{spacing}
